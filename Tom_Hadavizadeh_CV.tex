\documentclass[11pt,a4paper]{article}
\usepackage{array, xcolor, lipsum, bibentry}
\usepackage{titlesec}

\usepackage{ifthen}
\newboolean{pdflatex}
\setboolean{pdflatex}{false} % False for eps figures 

\newboolean{uprightparticles}
\setboolean{uprightparticles}{false} %Set true for upright particle symbols

\newboolean{inbibliography}
\setboolean{inbibliography}{false} %True once you enter the bibliography

\newboolean{articletitles}
\setboolean{articletitles}{true} % False removes titles in references

\include{lhcb-symbols-def}

\title{\bfseries\Huge Tom Hadavizadeh}
\author{tomhadavizadeh@gmail.com}
\usepackage[margin=1in]{geometry}

\definecolor{lightgray}{gray}{0.8}
\newcolumntype{L}{>{\raggedright}p{0.16\textwidth}}
%\newcolumntype{L}{>{\raggedright}p{0.20\textwidth}}
\newcolumntype{R}{p{0.8\textwidth}}
\newcommand\VRule{\color{lightgray}\vrule width 0.5pt}

%\newcolumntype{I}{>{\raggedleft}p{0.16\textwidth}}
\newcolumntype{I}{>{\raggedleft\arraybackslash}p{0.16\textwidth}}
\newcolumntype{E}{p{0.8\textwidth}}
%\newcolumntype{E}{>{\raggedright}p{0.8\textwidth}}
\newcolumntype{S}{>{\raggedright}p{0.4\textwidth}}
\newcolumntype{T}{>{\raggedright}p{0.38\textwidth}}
 

\titleformat{\section}
  {\normalfont\Large\bfseries}{\thesection}{1em}{}[{\color{lightgray}{\titlerule[0.8pt]}}]

%\usepackage{hyperref}
% \usepackage{cite}
% \usepackage{mciteplus}

% Make this the last packages you include before the \begin{document}
\usepackage{cite} % Allows for ranges in citations
\usepackage{mciteplus}

\begin{document}

\begin{minipage}[ht]{0.70\textwidth}
%\maketitle
{\bfseries\Huge Tom Hadavizadeh}\\[10pt]
% {\Large tom.hadavizadeh@physics.ox.ac.uk}\\
{\Large tomhadavizadeh@gmail.com}\\


\end{minipage}
\begin{minipage}[ht]{0.30\textwidth}
{\raggedleft
42 Tree Lane\\
Oxford\\
OX4 4EY\\
United Kingdom\\
+447847401189\\}
\end{minipage}


\section*{Research Interests}
I am a post-doctoral research assistant at the University of Oxford. I work on the proposed TORCH detector, a time-of-flight particle identification detector for low momentum charged hadrons. I have extensive experience performing data analysis at the \lhcb experiment, both during my DPhil and as a post-doctoral research assistant. During the latter I have broadened my skill set, becoming familiar with data acquisition (DAQ) software and the TORCH detector's electronics, as well as taking up teaching responsibilities. My research interests include making \CP violation measurements in beauty and charm hadrons and performing searches for rare decays. I have experience performing analyses with calorimeter objects (\eg electrons, photons, neutral pions) at \lhcb that I acquired from both my own analysis and as a Calorimeter Objects Liaison for the LHCb Charm physics working group.    


% My responsibilities include developing data acquisition software used in beam tests and laboratory measurements. 
% Additionally, I perform simulation studies to determine the potential performance improvements TORCH could provide to the \lhcb physic programme in the Upgrade IB and Upgrade II. I coordinated the complication of an internal LHCb document summarising the progress of the TORCH R\&D and simulation, and detailing the physics case.
% I have participated in the beam test campaigns, in both the commissioning of the hardware setup and participating in the data taking, and I am involved in the subsequent data analysis.
 % Alongside my TORCH responsibilities, I still maintain an active contribution to the LHCb experiment. I perform research into \CP asymmetries in two-body charm meson decays with neutral particles. These decays, for example \decay{\Dp}{\pip\piz}, are reconstructing using \decay{\piz}{\ep\en\gamma} decays, allowing measurements to be made that would otherwise have been out of reach at a hadron collider.      
% I am a fourth year DPhil student researching with the LHCb experiment and I am interested in pursuing a career in experimental particle physics. My DPhil thesis is centred around searches for rare annihilation topology decays of $B$ mesons, as these could potentially be sensitive to physics beyond the Standard Model. Alongside this I have gained a wide range of skills and experience, including data analysis, software development and in the running of the LHCb experiment. I hope to use these skills to tackle new and more complicated challenges. 

\section*{Research Experience and Education}
\noindent

\noindent\begin{tabular}{E!{\VRule} I }
{\bf Post-doctoral Research Assistant}   & {\bf 2018--present}\\
{\bf University of Oxford}&\\
\end{tabular}


\noindent
\begin{itemize}
\setlength\itemsep{0em}
\item \textbf{TORCH physics studies:} performed simulation studies to quantify the performance improvements from TORCH in the \lhcb Upgrade IB and beyond. I coordinated the compilation of an internal note documenting the progress in R\&D, simulation and physics studies (\texttt{LHCb-INT-2019-006}).
\item \textbf{TORCH DAQ:} developed and maintained Labview DAQ software used in beam tests and laboratory measurements.
\item \textbf{TORCH beam tests:} participated in the commissioning and running of two beam test campaigns.  
\item \textbf{LHCb reviewer:} acting as an internal reviewer for LHCb Charm physics analysis.
\item \textbf{LHCb analysis:} performing a measurement of \CP asymmetries in two-body charm meson decays with neutral particles. These decays, for example \decay{\Dp}{\pip\piz}, are reconstructing using \decay{\piz}{\ep\en\gamma} decays.
\item \textbf{Teaching:} tutor for 3rd year undergraduate Quantum, Atomic and Molecular physics course, and demonstrator for Nuclear Physics labs. \\[5pt]
\end{itemize}

\noindent\begin{tabular}{E!{\VRule} I }
{\bf DPhil in Particle Physics}   & {\bf 2014--2018} \\
{\bf St Peter's College, University of Oxford.} &\\
Supervisor: Malcolm John &\\
% ~~~~{\bf Thesis Title:} \emph{Rare hadronic charged B meson decays at LHCb} & \\
~~~~\emph{Rare hadronic decays of charged B mesons at LHCb} & \\
\texttt{CERN-THESIS-2018-290}& \\
% ~~~~{\bf Submission date:} April 2018 & \\
% ~~~~{\bf Long Term Attachment at CERN:} October 2015--March 2017& \\[5pt]
\end{tabular}



% \noindent\emph{Research Areas:} 
\begin{itemize}
\setlength\itemsep{0em}
\item {\bf $B^{+}$ meson decays:} Analysed LHCb data taken in Run I and II  to search for the rare decays $B^{+} \to D_{s}^{+} \phi$ and $B^{+} \to D_{s}^{+} K^{+} K^{-}$. In the standard model the decay $B^{+} \to D_{s}^{+} \phi$ is dominated by a single annihilation topology diagram, however New Physics could enhance the branching fraction. \textit{(published in JHEP)}

% \item {\bf $B_{c}^{+}$ meson decays:} Currently performing a search for $B_{c}^{+} \to D_{s}^{(*)+}\phi$  and $B_{c}^{+} \to D^{(*)+}K^{*0}$ decays. This is an important follow up to the recently published $B_{c}^{+} \to D^{0} K^{+}$ observation at LHCb, thought to be dominated by annihilation and penguin topology processes.

\item {\bf Real-time beam size measurements:} Helped develop and maintain an online monitoring task that measures the LHC beam sizes via Beam-Gas interactions within LHCb's VELO detector. A dedicated LHC-wide cross-calibration was performed. \textit{(poster at IPAC'17)}

\item {\bf Developed charm triggers in Run II:} Wrote software triggers dedicated to selecting $D^{+} \to \pi^{+} \pi^{0}$ and similar topology decays. These are reconstructed using $\pi^{0} \to e^{+} e^{-} \gamma$ decays and are being used to produce a \CP asymmetry measurement with the full Run I and~II dataset.
\end{itemize}

\noindent\emph{Collaboration Responsibilities:} 
\begin{itemize}
\setlength\itemsep{0em}
\item {\bf Charm working group CaloObjects liaison: (3 years)} reported to Charm WG about relevant topics and issues affecting calorimeter objects.  
% \begin{itemize}\setlength\itemsep{0em}
% \item Report to the Charm WG about relevant topics and issues affecting calorimeter objects.
% \item Perform studies for the CaloObjects group to validate and understand these particles.
% \end{itemize}

\item {\bf Control room shifts:} Performed 18 Shift Leader and 16 Data Manager shifts. \emph{(Ranked 3rd for most data manager shifts in 2016).} Responsible for the safety and quality of the LHCb experiment's running.
% \begin{itemize}
% \setlength\itemsep{0em}
% \item Performed 18 Shift Leader and 16 Data Manager shifts. \emph{(Ranked 3rd for most data manager shifts in 2016).}
% \item These roles are responsible for the safety and quality of the LHCb experiment's running. 
% \end{itemize}
\item {\bf LHCb UK Student Meeting committee: (2016 -- present)} Helped organise and run a series of talks aimed at PhD students.
% \begin{itemize}
% \setlength\itemsep{0em}
% \item Helped organise and run a series of talks aimed at new and current PhD students.
% \end{itemize}
\end{itemize}
~ 

\noindent\begin{tabular}{E!{\VRule} I }
{\bf Natural Sciences, MSci, MA, } First Class   & {\bf2010--2014}\\
{\bf Selwyn College, University of Cambridge} &\\
Part II and Part III subject: Experimental and Theoretical physics  &\\
\end{tabular}

% \noindent
% \begin{itemize}
% \setlength\itemsep{0em}
% \item Performed a study for the ATLAS collaboration, aiming to improve the selection of $ZZ \to \ell^{+}\ell^{-} \nu \bar{\nu}$ decays by exploiting multivariate techniques. 
% % \item This targeted  $ZZ \to \ell^{+}\ell^{-} \nu \bar{\nu}$ decays characterised by two leptons and a large missing energy. 
% % \item A number of multivariate techniques were tested to try and improve upon a cut-based approach.\\
% \end{itemize}
MSci Project: Performed a study for the ATLAS collaboration, aiming to improve the selection of $ZZ \to \ell^{+}\ell^{-} \nu \bar{\nu}$ decays by exploiting multivariate techniques.\\


% \begin{tabular}{L!{\VRule}R}
% Summer 2013     & {\bf DESY}\\
%                 & Hamburg, Germany\\[5pt]
% \end{tabular}
\noindent\begin{tabular}{E!{\VRule} I }
{\bf DESY Summer Studentship}   & {\bf 2013}\\
{\bf Hamburg, Germany}&\\
\end{tabular}

% \noindent
% \begin{itemize}
% \setlength\itemsep{0em}
% % \item Participated in the summer student programme at DESY in Hamburg for 8 weeks where I attended lectures on particle physics and completed a project with the FLC group.
% \item Performed simulation studies for the measurement of the Higgs Boson total decay width at the proposed International Linear Collider using C++ and ROOT. 
% % \item Presented my research to the FLC group and the summer students. \\
% \end{itemize}
Performed simulation studies to determine the potential precision of the Higgs Boson total decay width measurement at the proposed International Linear Collider. \\


\noindent\begin{tabular}{E!{\VRule} I }
{\bf RAL Summer Studentship}   & {\bf 2012 }\\
{\bf Harwell, Oxfordshire}&\\[5pt]
\end{tabular}
% \noindent 
% \begin{itemize}
% \setlength\itemsep{0em}
% % \item Completed an 8-week studentship at the Rutherford Appleton Labs working with the L1Calo group as part of the ATLAS collaboration. 
% % \item Created an algorithm to help improve the selectivity for the proposed upgrade of the hardware that decides which of the collisions in the LHC to record. My C++ algorithm identified energy deposits resulting from the decay of neutral pions into two photons and prevented these from initiating the trigger algorithm. 
% % \item Presented my progress to the RAL ATLAS group as well as the UK L1Calo group.
% \item Worked with the ATLAS collaboration's L1Calo group to create an algorithm to help improve the selectivity for the proposed upgrade of the hardware trigger by vetoing neutral pions.
% \end{itemize}
Worked with the ATLAS collaboration's L1Calo group to create an algorithm to help improve the selectivity for the proposed upgrade of the hardware trigger by vetoing neutral pions.\\


\section*{Relevant Research and Engagement Skills}

\begin{tabular}{E!{\VRule} I}
Proficient in C++, Python, LabView, Latex, ROOT, and GIT. I have developed LabView DAQ software that controls hardware using Ethernet, Serial and GPIB protocols. I have experience using machine learning algorithms in data analysis. & {\bf Computational} \\
&\\
Participated in the Cheltenham Science Fair, IF science festival, Stargazing Oxford, Super Science Saturday (Oxford Natural History museum), Conference for Undergraduate Women in Physics (Oxford). \emph{(2015--present)} & {\bf Outreach} \\
&\\
French (GCSE level and European A2.1), German (GCSE level). & {\bf Language} \\
\end{tabular}

\newpage
\section*{Awards and Prizes}

\noindent\begin{tabular}{S T!{\VRule}I }
% {\bf STFC PhD Studentship}           &                              & {\bf 2014--Present}\\
{\bf Selwyn College Braybrook Prize} & Best result in Part II       & {\bf June 2013} \\  
{\bf Selwyn College Scruby Prize}    & Best result in Part 1A or 1B & {\bf June 2012}\\
\end{tabular}



\setboolean{inbibliography}{true}
%\bibliography{main,LHCb-PAPER,LHCb-CONF,LHCb-DP,LHCb-TDR}
\nobibliography{LHCb-PAPER,publication}
%\bibliographystyle{LHCb}
%\bibliographystyle{unsrt}
% \bibliographystyle{LHCb}
% %\bibliographystyle{plain}
\bibliographystyle{hep}

% \newpage
\section*{Publications and Documents}

\begin{tabular}{E!{\VRule} I}
% \bibentry{LHCb-PAPER-2017-032}& \textbf{Main Author}\\

LHCb collaboration, \textit{First observation of $\Bp\to \Dsp K^+ K^-$ decays and a search for $\Bp\to \Dsp\phi$ decays}, JHEP \textbf{01}, 131
(2018), 1711.05637, \texttt{LHCb-PAPER-2017-032} & \textbf{Main Author}\\
~~~~~~\textit{- Performed all aspects of the analysis with advice from Malcolm John and Laurence Carson.}\\[15pt]

Member of LHCb collaboration author list since Summer 2015 ($\sim$ 200 publications on arXiv) & \textbf{Collaborator} \\[25pt]

LHCb internal note, \textit{TORCH physics performance: improving low-momentum PID performance during Upgrade IB and beyond}, \texttt{LHCb-INT-2019-006} & \textbf{Author and coordinator}\\
~~~~~~\textit{- Performed all physics studies and coordinated the document compilation by convening dedicated meetings. }\\[15pt]
\end{tabular}

% \newpage
\section*{Conference Proceedings}

\begin{tabular}{E!{\VRule} I}
% \bibentry{Hadavizadeh:IPAC2017-MOPAB131}& \textbf{Author and Presenter \emph{(poster)}} \\[35pt]
% \bibentry{Alemany-Fernandez:IPAC2017-MOPAB130} & \textbf{Author}\\[75pt]
% \bibentry{Hostettler:IPAC2017-MOPAB110} & \textbf{Author} \\[5pt]
T. Hadavizadeh et al., \textit{Transverse Emittance Measurements Using LHCb's Beam-Gas Interactions}, IPAC'17, {doi.org/10.18429/JACoW-IPAC2017-MOPAB131} & \textbf{First author} \\
~~~~~~\textit{- Created and presented poster}\\[5pt]

N. Harnew et al., \textit{TORCH: a large area time-of-flight detector for particle identification}, 13th Pisa Meeting on Advanced Detectors, {doi.org/10.1016/j.nima.2018.10.099} & \textbf{Author}\\
~~~~~~\textit{- Created poster}\\[5pt]

M. Hostettler et al., \textit{Comparison of Transverse Emittance Measurements in the LHC} IPAC'17, {doi.org/10.18429/JACoW-IPAC2017-MOPAB110} & \textbf{Author} \\
~~~~~~\textit{- Contributed to analysis}\\[5pt]

R. Alemany-Fernndez et al., \textit{Cross-Calibration of the LHC Transverse Beam-Profile Monitors}, IPAC'17, {doi.org/10.18429/JACoW-IPAC2017-MOPAB130} & \textbf{Author}\\
~~~~~~\textit{- Contributed to analysis}\\[5pt]

N. Harnew et al., \textit{Status of the TORCH time-of-flight project}, RICH 2018, {arXiv:1812.09773} & \textbf{Author}\\
~~~~~~\textit{- Contributed to beam tests}\\[5pt]
\end{tabular}



\newpage
\section*{Notable Talks and Presentation}
\begin{tabular}{E!{\VRule} I}
\textbf{International Conference:} Lake Louise Winter Institute \textit{Very Rare B hadron decays}      & \emph{Feb 2018}   \\[3pt]
\textbf{National Conference:} Institute of Physics Joint APP and HEPP Annual Conference \textit{The \decay{\Bp}{\Dsp\phiz} analysis} & \emph{Apr 2017}\\[3pt] 
\textbf{Seminar:} St Peter's College Graduate Seminar \textit{Where's all the antimatter gone?}         & \emph{May 2017}   \\[3pt]
\textbf{Lecture:} Oxford PP summer school \textit{Introduction to Heavy flavour physics}                & \emph{Jul 2018}   \\[3pt]
\textbf{Workshop:} LHCb Upgrade II workshop \textit{The TORCH detector; physics case}                   & \emph{Apr 2019}   \\[3pt]
\textbf{Internal:} TORCH collaboration meetings \textit{DAQ and physics studies}                        & \emph{2018--2019} \\[3pt]
\textbf{Internal:} LHCb Week \textit{The TORCH detector; recent results}                                & \emph{Dec 2018}   \\[3pt]
\textbf{Internal:} LHCb Week \textit{Summary of PID\&CO group}                                          & \emph{Dec 2017}   \\[3pt]
\textbf{Internal:} LHCb UK meeting \textit{\decay{\Bp}{\Dsp\phiz}; An annihilation topology decay}      & \emph{Jan 2017}   \\[3pt] 
\textbf{Internal:} LHCb Analysis and software week \textit{Status of \decay{\Bp}{\Dsp\phiz} analysis}   & \emph{Jul 2016}   \\[3pt]

\end{tabular}



% \newpage
\section*{Teaching Experience}
\begin{tabular}{E!{\VRule}I}

\textbf{Tutor} in Quantum, Atomic and Molecular physics (3rd year UG) (Jesus College)    & \textbf{2018--present}\\[5pt]
\textbf{Demonstrator} in Nuclear Physics lab (3rd year UG)                               & \textbf{2018--present}\\[5pt]
\textbf{Lecturer} for Oxford PP summer school - \textit{Introduction to Heavy flavour physics}    & {\bf Jul 2018}\\[5pt]
\textbf{Effectively supervised} a summer student; I helped design a project optimising the selection of $D^{+} \to \pi^{+} \pi^{0}$ decays and provided help and support throughout the summer.& {\bf 2017} \\[5pt]
\textbf{Assisted in the supervision} of a summer/masters student. We collaboratively studied two selection techniques and compared the results. & {\bf 2015} \\ 

\end{tabular}


% \section*{Education}

% \begin{tabular}{E!{\VRule} I}
% {\bf St Peter's College, University of Oxford,} DPhil in Particle Physics     & {\bf 2014--2018} \\[5pt]
% Thesis Title: \emph{Rare hadronic decays of charged B mesons at LHCb}         & \\[10pt]
% {\bf Selwyn College, University of Cambridge,} Natural Sciences, MSci, MA & {\bf 2010--2014}\\[5pt]
% MSci Project: \emph{Optimizing the selection of ZZ events at the LHC (ATLAS)} & \\[5pt]
% \begin{tabular}{ l l l }
% {\bf Part III:} & Experimental and Theoretical Physics        & {\bf First Class} 78\%\\
% {\bf Part II:} & Experimental and Theoretical Physics         & {\bf First Class} 85\%\\
% {\bf Part IB:} & Physics and Physical Chemistry               & {\bf First Class} 82\%\\
% {\bf Part IA:} & Biology of Cells, Chemistry, Physics, Maths  & {\bf First Class} 76\%\\[10pt]
% \end{tabular} & \\[10pt]
%  {\bf St Laurence School, Bradford on Avon} & {\bf 2003--2010}\\[5pt]
%  \begin{tabular}{ l l }
% {\bf A Levels:}  & Physics (A*), Maths (A*), Chemistry (A*)\\
% {\bf As Levels:} & Geography (A), Further Maths (A)\\
% {\bf GCSE Level:}& 12 A*, 1A, 1B (including Maths and English)\\
% \end{tabular}& \\[10pt]
% \end{tabular}


% \newpage
\section*{Referees}

\indent\textit{Current supervisor}\\
\textbf{Professor Neville Harnew}, University of Oxford\\
Head of Oxford LHCb group, TORCH project leader\\
\texttt{neville.harnew@physics.ox.ac.uk}\\[5pt]

\noindent\textit{Former LHCb Spokesperson and TORCH collaborator}\\
\noindent \textbf{Professor Guy Wilkinson}, University of Oxford\\
\texttt{Guy.Wilkinson@cern.ch}\\[5pt]

\noindent\textit{Deputy Head of CERN Physics department and TORCH collaborator}\\
\noindent \textbf{Roger Forty}, CERN, Geneva, CH, \\
\texttt{Roger.Forty@cern.ch}\\[5pt]



% {\bf DPhil Supervisor}, Malcolm John, Department of Physics, Oxford, UK,\\ 
% \texttt{malcolm.john@physics.ox.ac.uk}.\\[5pt]
% {\bf LHCb Luminosity Group}, Massimiliano Ferro-Luzzi, CERN, Geneva, CH, \\
% \texttt{Massimiliano.Ferro-Luzzi@cern.ch}\\[5pt]
% {\bf LHCb Physics Coordinator}, Vincenzo Vagnoni, CERN, Geneva, CH, \\
% \texttt{vincenzo.vagnoni@cern.ch}\\

% \section*{Status of the TORCH time-of-flight detector}

% TORCH is a novel time-of-flight detector, designed to provide $\pi$/K particle identification up to 10 GeV/$c$ momentum over a 10 m flight path. Based on the DIRC principle, Cherenkov photons are produced in a quartz plate of 10 mm thickness, where they propagate to the periphery of the plate by total-internal reflection. There the photons are focused onto an array of micro-channel plate photomultipliers (MCP-PMTs) which measure their arrival times and spatial positions. A time resolution of 70 ps per detected Cherenkov photon is expected, which results in a time-of-flight resolution of 15 ps, given typically 30 detected photons per track. To demonstrate the principle, a half-scale ($660\times1250\times10$ mm$^3$) TORCH prototype module has been tested in a 5 GeV/$c$ mixed proton-pion beam at the CERN PS. Customised $53\times53$ mm$^2$ MCP-PMTs of effective granularity $128\times8$ pixels have been employed, which have been developed in collaboration with an industrial partner. The single-photon timing performance and photon yields have been measured and are close to specification, demonstrating the TORCH concept. For a future application, a full-scale TORCH detector has been proposed for the LHCb Phase II Upgrade, which comprises 18 modules with 198 MCP-PMTs. Results will be reported on the simulated performance of the detector for high luminosity LHCb running in terms of $\pi$/K/p discrimination.




\end{document}

