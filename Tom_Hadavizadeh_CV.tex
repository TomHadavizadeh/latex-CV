\documentclass[11pt,a4paper]{article}
\usepackage{array, xcolor, lipsum, bibentry}
\usepackage{titlesec}

\usepackage{ifthen}
\newboolean{pdflatex}
\setboolean{pdflatex}{false} % False for eps figures 

\newboolean{uprightparticles}
\setboolean{uprightparticles}{false} %Set true for upright particle symbols

\newboolean{inbibliography}
\setboolean{inbibliography}{false} %True once you enter the bibliography

\newboolean{articletitles}
\setboolean{articletitles}{true} % False removes titles in references

\include{lhcb-symbols-def}

\title{\bfseries\Huge Tom Hadavizadeh}
\author{tomhadavizadeh@gmail.com}
\usepackage[margin=1in]{geometry}

\definecolor{lightgray}{gray}{0.8}
\newcolumntype{L}{>{\raggedright}p{0.16\textwidth}}
%\newcolumntype{L}{>{\raggedright}p{0.20\textwidth}}
\newcolumntype{R}{p{0.8\textwidth}}
\newcommand\VRule{\color{lightgray}\vrule width 0.5pt}

%\newcolumntype{I}{>{\raggedleft}p{0.16\textwidth}}
\newcolumntype{I}{>{\raggedleft\arraybackslash}p{0.16\textwidth}}
\newcolumntype{E}{p{0.8\textwidth}}
%\newcolumntype{E}{>{\raggedright}p{0.8\textwidth}}
\newcolumntype{S}{>{\raggedright}p{0.4\textwidth}}
\newcolumntype{T}{>{\raggedright}p{0.38\textwidth}}
 

\titleformat{\section}
  {\normalfont\Large\bfseries}{\thesection}{1em}{}[{\color{lightgray}{\titlerule[0.8pt]}}]

%\usepackage{hyperref}
% \usepackage{cite}
% \usepackage{mciteplus}

% Make this the last packages you include before the \begin{document}
\usepackage{cite} % Allows for ranges in citations
\usepackage{mciteplus}

\begin{document}

\begin{minipage}[ht]{0.70\textwidth}
%\maketitle
{\bfseries\Huge Tom Hadavizadeh}\\[10pt]
% {\Large tom.hadavizadeh@physics.ox.ac.uk}\\
{\Large tomhadavizadeh@gmail.com}\\


\end{minipage}
\begin{minipage}[ht]{0.30\textwidth}
{\raggedleft
42 Tree Lane\\
Oxford\\
OX4 4EY\\
United Kingdom\\
+447847401189\\}
\end{minipage}


\section*{Research Interests}
I am a post-doctoral research assistant at the University of Oxford. I work mainly on the proposed TORCH detector, a time-of-flight particle identification detector for low momentum charged hadrons. My responsibilities include developing data acquisition software used in beam tests and laboratory measurements. Additionally, I perform simulation studies to determine the potential performance improvements TORCH could provide to the \lhcb physic programme in the Upgrade IB and Upgrade II. I coordinated the complication of an internal LHCb document summarising the progress of the TORCH R\&D and simulation, and detailing the physics case.
 I have participated in the beam test campaigns, in both the commissioning of the hardware setup and participating in the data taking, and I am involved in the subsequent data analysis.

 Alongside my TORCH responsibilities, I still maintain an active contribution to the LHCb experiment. I perform research into \CP asymmetries in two-body charm meson decays with neutral particles. These decays, for example \decay{\Dp}{\pip\piz}, are reconstructing using \decay{\piz}{\ep\en\gamma} decays, allowing measurements to be made that would otherwise have been out of reach at a hadron collider.      


% I am a fourth year DPhil student researching with the LHCb experiment and I am interested in pursuing a career in experimental particle physics. My DPhil thesis is centred around searches for rare annihilation topology decays of $B$ mesons, as these could potentially be sensitive to physics beyond the Standard Model. Alongside this I have gained a wide range of skills and experience, including data analysis, software development and in the running of the LHCb experiment. I hope to use these skills to tackle new and more complicated challenges. 

\section*{Research Experience}
\noindent

\noindent\begin{tabular}{E!{\VRule} I }
{\bf Post-doctoral Research Assistant}   & {\bf 2018--present}\\
{\bf University of Oxford}&\\
\end{tabular}


\noindent
\begin{itemize}
\setlength\itemsep{0em}
\item TORCH DAQ
\item TORCH test beam data taking
\item TORCH physics simulations
\item LHCb reviewer
\item LHCb analysis
\item Teaching
\end{itemize}

\noindent\begin{tabular}{E!{\VRule} I }
{\bf DPhil Research}   & {\bf 2014--2018} \\
{\bf Department of Physics, University of Oxford}&\\[5pt]
~~~~{\bf Thesis Title:} \emph{Rare hadronic charged B meson decays at LHCb} & \\
% ~~~~{\bf Submission date:} April 2018 & \\
~~~~{\bf Long Term Attachment at CERN:} October 2015--March 2017& \\[5pt]

\end{tabular}



\noindent\emph{Research Areas:} 
\begin{itemize}
\setlength\itemsep{0em}
\item {\bf $B^{+}$ meson decays:} Analysed LHCb data taken in Run I and Run II  to search for the rare decays $B^{+} \to D_{s}^{+} \phi$ and $B^{+} \to D_{s}^{+} K^{+} K^{-}$. In the standard model the decay $B^{+} \to D_{s}^{+} \phi$ is dominated by a single annihilation topology diagram, however a number of theories beyond the Standard Model could enhance the branching fraction. The resulting paper has been submitted to JHEP.

% \item {\bf $B_{c}^{+}$ meson decays:} Currently performing a search for $B_{c}^{+} \to D_{s}^{(*)+}\phi$  and $B_{c}^{+} \to D^{(*)+}K^{*0}$ decays. This is an important follow up to the recently published $B_{c}^{+} \to D^{0} K^{+}$ observation at LHCb, thought to be dominated by annihilation and penguin topology processes.

\item {\bf Real-time beam size measurements:} Helped develop and maintain an online monitoring task that measures the LHC beam sizes via Beam-Gas interactions with LHCb's VELO detector. A dedicated cross-calibration was performed and the results present as a poster at IPAC 2017.

\item {\bf Developed charm triggers in Run II:} Wrote software triggers dedicated to selecting $D^{+} \to \pi^{+} \pi^{0}$ and similar topology decays. These are reconstructed using $\pi^{0} \to e^{+} e^{-} \gamma$ decays and will be used to produce a CP asymmetry measurement with the Run~II dataset.
\end{itemize}

\noindent\emph{Collaboration Responsibilities:} 
\begin{itemize}
\setlength\itemsep{0em}
\item {\bf Charm working group CaloObjects liaison: (3 years)} reported to Charm WG about relevant topics and issues affecting calorimeter objects.  
% \begin{itemize}\setlength\itemsep{0em}
% \item Report to the Charm WG about relevant topics and issues affecting calorimeter objects.
% \item Perform studies for the CaloObjects group to validate and understand these particles.
% \end{itemize}

\item {\bf Control room shifts:} Performed 18 Shift Leader and 16 Data Manager shifts. \emph{(Ranked 3rd for most data manager shifts in 2016).} Responsible for the safety and quality of the LHCb experiment's running.
% \begin{itemize}
% \setlength\itemsep{0em}
% \item Performed 18 Shift Leader and 16 Data Manager shifts. \emph{(Ranked 3rd for most data manager shifts in 2016).}
% \item These roles are responsible for the safety and quality of the LHCb experiment's running. 
% \end{itemize}
\item {\bf LHCb UK Student Meeting committee: (201X -- present)} Helped organise and run a series of talks aimed at new and current PhD students.
% \begin{itemize}
% \setlength\itemsep{0em}
% \item Helped organise and run a series of talks aimed at new and current PhD students.
% \end{itemize}
\end{itemize}
~ 

\noindent\begin{tabular}{E!{\VRule} I }
{\bf MSci Project}   & {\bf2013--2014}\\
{\bf Cavendish Laboratory, University of Cambridge}&\\
\end{tabular}

\noindent
\begin{itemize}
\setlength\itemsep{0em}
\item Performed a study with the ATLAS experiment, aiming to improve the selection of ZZ events by exploiting multivariate techniques. 
\item This targeted $ZZ \to \ell^{+}\ell^{-} \nu \bar{\nu}$ decays characterised by two leptons and a large missing energy. 
\item A number of multivariate techniques were tested to try and improve upon a cut-based approach.\\
\end{itemize}



% \begin{tabular}{L!{\VRule}R}
% Summer 2013     & {\bf DESY}\\
%                 & Hamburg, Germany\\[5pt]
% \end{tabular}
\noindent\begin{tabular}{E!{\VRule} I }
{\bf DESY Summer Studentship}   & {\bf Summer 2013}\\
{\bf Hamburg, Germany}&\\
\end{tabular}

\noindent
\begin{itemize}
\setlength\itemsep{0em}
\item Participated in the summer student programme at DESY in Hamburg for 8 weeks where I attended lectures on particle physics and completed a project with the FLC group.
\item Performed simulation studies for the measurement of the Higgs Boson total decay width at the proposed International Linear Collider using C++ and ROOT. 
\item Presented my research to the FLC group and the summer students. \\
\end{itemize}


\noindent\begin{tabular}{E!{\VRule} I }
{\bf RAL Summer Studentship}   & {\bf Summer 2012 }\\
{\bf Harwell, Oxfordshire}&\\[5pt]
\end{tabular}
\noindent 
\begin{itemize}
\setlength\itemsep{0em}
\item Completed an 8-week studentship at the Rutherford Appleton Labs working with the L1Calo group as part of the ATLAS collaboration. 
\item Created an algorithm to help improve the selectivity for the proposed upgrade of the hardware that decides which of the collisions in the LHC to record. My C++ algorithm identified energy deposits resulting from the decay of neutral pions into two photons and prevented these from initiating the trigger algorithm. 
\item Presented my progress to the RAL ATLAS group as well as the UK L1Calo group.
\end{itemize}


% \begin{tabular}{L!{\VRule}R}
% DPhil Research  & \emph{Research Areas:} \begin{itemize}
%                   \setlength\itemsep{0em}
%                   \item {\bf Rare hadronic decays of charged B mesons:} 
%                   \item {\bf Real-time beam size measurements:} I helped develop and maintain an online monitoring task that measures the LHC beam sizes via Beam-Gas interactions with LHCb's VELO detector. A dedicated cross-calibration was performed and the results present as a poster at IPAC 2017.
%                   \item {\bf Developed charm triggers in Run II:} I made software triggers to select $D^{+} \to \pi^{+} \pi^{0}$ and similar topology decays. These are reconstructed using $\pi^{0} \to e^{+} e^{-} \gamma$ decays.
%                   \end{itemize}       \\

%                 & \emph{Responsibilities:} \begin{itemize}
%                   \setlength\itemsep{0em}
%                   \item {\bf Charm Working Group CaloObjects Liaison: } I report to the Charm WG about relevant topics and issues affecting calorimeter objects. I perform studies for the CaloObjects group to help validate and understand these particles.
%                   \item {\bf Shift Leader and Data Manager:} I have performed X control room shifts in which I have been responsible for monitoring the safety and quality of the LHCb experiment's running.
%                   \item {\bf LHCb UK Student Meeting Committee:} I help organise and run a series of talks aimed at PhD students. 
%                   \end{itemize}       \\[5pt]
                
% MSci Project    & I performed a study with the ATLAS experiment, aiming to improve the selection of ZZ events by exploiting multivariate techniques. This targeted $ZZ \to \ell^{+}\ell^{-} \nu \bar{\nu}$ decays characterised by two leptons and a large missing energy. A number of multivariate techniques were tested to try and improve upon a cut-based approach. \emph{(2013-2014)}\\[5pt]

% DESY    & I participated in the summer student programme at DESY in Hamburg for 8 weeks where I attended lectures on particle physics and completed a project with the FLC group. Here I performed simulation studies for the measurement of the Higgs Boson total decay width at the proposed International Linear Collider using C++ and ROOT. I presented my research to the FLC group and the summer students. \emph{(Summer 2013)}          \\[5pt]
% \end{tabular}\\


% \begin{tabular}{L!{\VRule}R}
% RAL     & I completed an 8-week studentship at the Rutherford Appleton Labs working with the L1Calo group as part of the ATLAS collaboration. I created an algorithm to help improve the selectivity for the proposed upgrade of the hardware that decides which of the collisions in the LHC to record. My C++ algorithm identified energy deposits resulting from the decay of neutral pions into two photons and prevented these from initiating the trigger algorithm. I presented my progress to the RAL ATLAS group as well as the UK L1Calo group. \emph{(Summer 2012)}         \\[5pt]

% \end{tabular}
\newpage
\section*{Relevant Research and Engagement Skills}

\begin{tabular}{E!{\VRule} I}
Proficient in C++, Python, LabView, Latex, ROOT, and GIT. I have written LabView DAQ software that controls hardware using Ethernet, Serial and GPIB protocols. & {\bf Computational} \\[10pt]
 
% I have communicated my research to collaborators in a number of internal and external presentations. This includes the annual LHCb UK meeting \emph{(Jan 2017)} and Institute of Physics Joint APP and HEPP Annual Conference \emph{(Apr 2017)}. I have also presented to a general audience at a St Peter's College Graduate Seminar \emph{(May 2017)}. & {\bf Presentations} \\[60pt]

Participated in the Cheltenham Science Fair, IF science festival, Stargazing Oxford, Super Science Saturday (Oxford Natural History museum), Conference for Undergraduate Women in Physics (Oxford). \emph{(2015--present)} & {\bf Outreach} \\[10pt]
French (GCSE level and European A2.1), German (GCSE level). & {\bf Language} \\
\end{tabular}

\section*{Talks and presentation}

\begin{tabular}{E!{\VRule} I}
\textbf{Workshop:} LHCb Upgrade II workshop \textit{TORCH physics case} & \emph{(Apr 2019)}\\[3pt]
\textbf{Internal:} TORCH meetings (ox,CERN,Edinburgh)  & \emph{(Feb 2018)}\\[3pt]
\textbf{Internal:} LHCb week \textit{TORCH overview} & \emph{(Feb 2018)}\\[3pt]
\textbf{Conference:} Lake Louise Winter Institute \textit{Very Rare B hadron decays} & \emph{(Feb 2018)}\\[3pt]



\textbf{Internal} LHCb UK meeting \textit{\decay{\Bp}{\Dsp\phiz} analysis} & \emph{(Jan 2017)} \\[3pt] 
\textbf{National} Institute of Physics Joint APP and HEPP Annual Conference \textit{\decay{\Bp}{\Dsp\phiz} analysis} & \emph{(Apr 2017)}\\[3pt] 
\textbf{Seminar} St Peter's College Graduate Seminar \textit{Where's all the antimatter gone?}  & \emph{(May 2017)}\\[3pt]
\textbf{Internal} LHCb Week \textit{Summary of PID group} & \emph{(Feb 2018)}\\[3pt]
\textbf{Internal} LHCb Analysis and software week \textit{\decay{\Bp}{\Dsp\phiz} analysis} & \emph{(Feb 2018)}\\[3pt]

\end{tabular}


\section*{Awards and Prizes}

\noindent\begin{tabular}{S T!{\VRule}I }
% {\bf STFC PhD Studentship}           &                              & {\bf 2014--Present}\\
{\bf Selwyn College Braybrook Prize} & Best result in Part II       & {\bf June 2013} \\  
{\bf Selwyn College Scruby Prize}    & Best result in Part 1A or 1B & {\bf June 2012}\\
\end{tabular}


\setboolean{inbibliography}{true}
%\bibliography{main,LHCb-PAPER,LHCb-CONF,LHCb-DP,LHCb-TDR}
\nobibliography{LHCb-PAPER,publication}
%\bibliographystyle{LHCb}
%\bibliographystyle{unsrt}
% \bibliographystyle{LHCb}
% %\bibliographystyle{plain}
\bibliographystyle{hep}


\section*{Publications}

\begin{tabular}{E!{\VRule} I}
Member of LHCb collaboration author list since Summer 2015.& {\bf Collaborator} \\[5pt]
% \bibentry{LHCb-PAPER-2017-032}& {\bf Main Author}\\

LHCb collaboration, \textit{First observation of $\Bp\to \Ds K^+ K^-$ decays and a search for $\Bp\to \Ds\phi$ decays}, JHEP \textbf{01}, 131
(2018), 1711.05637, \texttt{LHCb-PAPER-2017-032} & {\bf Main Author}\\

\end{tabular}


\section*{Conference Proceedings}

\begin{tabular}{E!{\VRule} I}
% \bibentry{Hadavizadeh:IPAC2017-MOPAB131}& {\bf Author and Presenter \emph{(poster)}} \\[35pt]
% \bibentry{Alemany-Fernandez:IPAC2017-MOPAB130} & {\bf Author}\\[75pt]
% \bibentry{Hostettler:IPAC2017-MOPAB110} & {\bf Author} \\[5pt]
T. Hadavizadeh et al., \textit{Transverse Emittance Measurements Using LHCb’s Beam-Gas Interactions}, IPAC’17, {doi.org/10.18429/JACoW-IPAC2017-MOPAB131} & {\bf Author and Presenter \emph{(poster)}} \\[5pt]
M. Hostettler et al., \textit{Comparison of Transverse Emittance Measurements in the LHC} IPAC’17, {doi.org/10.18429/JACoW-IPAC2017-MOPAB110} & {\bf Author} \\[5pt]

R. Alemany-Fernndez et al., \textit{Cross-Calibration of the LHC Transverse Beam-Profile Monitors}, IPAC’17, {doi.org/10.18429/JACoW-IPAC2017-MOPAB130} & {\bf Author}\\[5pt]

\end{tabular}


\section*{Teaching Experience}
\begin{tabular}{E!{\VRule}I}

Tutor in Quantum, Atomic and Molecular physics (3rd year UG) (Jesus College) & {\bf Oct 2018 - present}\\[5pt]
Demonstrator in Nuclear Physics lab (3rd year UG)  & {\bf Oct 2018 - present}\\[5pt]
Oxford PP summer school lecturer - Introduction to Heavy flavour physics   & {\bf Jul 2018}\\[5pt]
Effectively supervised a summer student; I helped design a project optimising the selection of $D^{+} \to \pi^{+} \pi^{0}$ decays and provided help and support throughout the summer.& {\bf Summer 2017} \\[5pt]
Assisted in the supervision of a summer/masters student. We collaboratively studied two selection techniques and compared the results. & {\bf Summer 2015} \\ 

\end{tabular}

\section*{Education}

\begin{tabular}{E!{\VRule} I}
{\bf St Peter's College, Oxford University} DPhil in Particle Physics     & {\bf 2014--2018} \\[5pt]
Thesis Title: \emph{Rare hadronic charged B meson decays at LHCb}         & \\[10pt]
{\bf Selwyn College, University of Cambridge,} Natural Sciences, MSci, MA & {\bf 2010--2014}\\[5pt]
Project Title: \emph{Optimizing the selection of ZZ events at the LHC (ATLAS)} & \\[5pt]
\begin{tabular}{ l l l }
{\bf Part III:} & Experimental and Theoretical Physics        & {\bf First Class} 78\%\\
{\bf Part II:} & Experimental and Theoretical Physics         & {\bf First Class} 85\%\\
{\bf Part IB:} & Physics and Physical Chemistry               & {\bf First Class} 82\%\\
{\bf Part IA:} & Biology of Cells, Chemistry, Physics, Maths  & {\bf First Class} 76\%\\[10pt]
\end{tabular} & \\[10pt]
 {\bf St Laurence School, Bradford on Avon} & {\bf 2003--2010}\\[5pt]
 \begin{tabular}{ l l }
{\bf A Levels:}  & Physics (A*), Maths (A*), Chemistry (A*)\\
{\bf As Levels:} & Geography (A), Further Maths (A)\\
{\bf GCSE Level:}& 12 A*, 1A, 1B (including Maths and English)\\
\end{tabular}& \\[10pt]
\end{tabular}


% \section*{Relevant Skills and Courses}

% \begin{tabular}{E!{\VRule} I }
% {\bf French Language:} Achieved European level A2.1 whilst on Long Term Attachment at CERN.& {\bf 2016}\\[20pt]
% {\bf STFC Summer school:} Completed courses covering a range of topics within particle physics. & {\bf 2015} \\[20pt]
% {\bf Oxford first year lectures:} Successfully completed courses during the first year of my DPhil. Including: QCD, QFT, Advanced Quantum Mechanics, Symmetries, Statistics and an introduction to Accelerator Physics. & {\bf 2014--2015}\\[5pt]
% \end{tabular}


\section*{Referees}

{\bf DPhil Supervisor}, Malcolm John, Department of Physics, Oxford, UK,\\ 
\texttt{malcolm.john@physics.ox.ac.uk}.\\[5pt]
{\bf LHCb Luminosity Group}, Massimiliano Ferro-Luzzi, CERN, Geneva, CH, \\
\texttt{Massimiliano.Ferro-Luzzi@cern.ch}\\[5pt]
{\bf LHCb Physics Coordinator}, Vincenzo Vagnoni, CERN, Geneva, CH, \\
\texttt{vincenzo.vagnoni@cern.ch}\\





\end{document}

